
\chapter{Blueprint}

    I blueprint di Unreal Engine permettono di programmare in modo visuale, attraverso una interfaccia grafica.

    Per ogni blueprint e' possibile vedere e modificare l'aspetto del blueprint dalla finestra "viewport" e modificare gli eventi gestiti dal blueprint in "event graph".


    % VIEWPORT
    \section{Viewport}
        In alto a sinistra e' possibile aggiungere componenti al blueprint cliccando sul tasto "Add".

        Alcuni componenti utili sono:
        \begin{itemize}
            \item "static mesh": e' l'aspetto che avra' l'oggetto all'interno del gioco
            \item "arrow" che permette ad esempio di ottenere la sua direzione ed e' solo visibile nel viewport principale e non in gioco
            \item "box collision" per aggiungere collisioni
        \end{itemize}


    % EVENT GRAPH
    \section{Event graph}

        In event graph e' possibile configurare cosa succede all'istanza del blueprint in base agli eventi che vengono lanciati.
        E' uno schema a blocchi in cui si specifica quali sequenze di azioni eseguire quando certi eventi sono attivi.

        Da questa finestra e' possibile aggiungere variabili sotto a "variables" (e possono essere modificate anche dall'editor nel "details panel")
        e aggiungere blocchi/nodi per eseguire funzioni quando viene lanciato un evento.

        I due eventi di default sono "Event tick" che viene eseguito ad ogni frame e "Event BeginPlay" che viene eseguito quando viene fatto partire il gioco.

        Cliccando con il tasto destro e' possibile cercare e piazzare nuovi blocchi/nodi.

        Alcuni blocchi utili sono:

        \begin{itemize}
            \item "Get Overlapping Actors" su una box collision: permette di recuperare tutti gli oggetti all'interno della collisione
            \item "For Each Loop" permette di effettuare un loop for each scorrendo tutti gli oggetti di una lista e per ogni elemento eseguire una serie di operazioni
            \item "Get Class" permette di recuperare la classe di un oggetto
            \item "Class is child of" permette di verificare insieme a "Get class" se un oggetto e' figlio di una classe
            \item "Branch" permette di eseguire il ramo true se una condizione e' vera oppure il ramo false se la condizione e' falsa
            \item "Add Actor World Offset" permette di spostare un oggetto di una quantita' specificata da un vettore (X, Y, Z) in input relativamente alla sua posizione attuale
            \item "Multiply" permette di moltiplicare due elementi insieme
            \item "Break Vector" permette di separare un vettore nelle sue 3 componenti float X, Y, Z
            \item "Make Vector" permette di costruire un vettore a partire da 3 componenti float X, Y, Z
            \item "Select Vector" permette di selezionare e restituire un vettore se una condizione e' vera oppure restituirne un altro se la condizione e' falsa.
        \end{itemize}


    % CONFRONTO CON CODICE C++
    \section{Confronto con codice C++}
        Vantaggi dei blueprint rispetto al codice C++:
        \begin{itemize}
            \item Con i blueprint, essendo un "linguaggio" che viene interpretato a runtime, non occorre attendere il tempo di compilazione che e' necessario prima di eseguire il codice C++.

                Questo significa che, in generale, i tempi di sviluppo sono piu' rapidi mediante i blueprint.
            \item Non e' necessario saper programmare con un linguaggio di programmazione "tradizionale" per aggiungere logica al gioco.
        \end{itemize}

        Svantaggi dei blueprint rispetto al codice C++:
        \begin{itemize}
            \item Essendo file binari non e' possibile sfruttare appieno gli strumenti di versioning control come git.

                I sorgenti scritti in codice C++, essendo file di testo, permettono, ad esempio, di vedere ad ogni commit quali porzioni di codice sono state modificate all'interno di un file sorgente.

                Con i blueprint e' solo possibile vedere che quel file e' stato modificato: per rimuovere una certa modifica bisogna sapere quale commit ha effettuato la modifica indesiderata e fare il "rollback" dell'intero blueprint (eliminando anche cio' che si desidererebbe mantenere).

            \item Le prestazioni dei blueprint, essendo interpretati a runtime, possono risultare peggiori rispetto a codice scritto in C++.
        \end{itemize}

        E' consigliato scrivere in codice C++:
        \begin{itemize}
            \item tutto cio' che deve avere alte prestazioni
            \item file sorgenti la cui scrittura richiede la collaborazione tra piu' programmatori, aiutandosi con strumenti come git
        \end{itemize}

        E' consigliato utilizzare i blueprint:
        \begin{itemize}
            \item per interfacciare porzioni di codice scritte internamente in C++
            \item per creare blueprint, possibilmente "piccoli" (logica implementata semplice, pochi nodi, \dots), che non richiedono alte prestazioni oppure per creare blueprint che vengono chiamati raramente
            \item per la configurazione di valori di parametri che possono variare spesso durante la fase di sviluppo
            \item per permettere a personale che svolge mansioni diverse dalla programmazione (come, ad esempio, i level designer) di configurare facilmente alcuni parametri o di sviluppare blueprint piu' legati alla user experience

                \begin{notebox}
                    Alcune proprieta' che possono essere esposte attraverso i blueprint sono la velocita' del giocatore, quanto e' alto un salto, \dots
                \end{notebox}
        \end{itemize}
