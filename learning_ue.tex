\documentclass{book}

% imposta encoding e lingua italiana
\usepackage[utf8]{inputenc}
\usepackage[T1]{fontenc}
\usepackage[italian]{babel}

% imposta dimensioni pagina
\usepackage[bottom=1.5cm, right=3.5cm, left=3.5cm, top=2.5cm]{geometry}

\usepackage{import}     % per suddividere la documentazione su piu' file .tex
\usepackage{emptypage}  % per rimuovere numeri di pagina dalle pagine vuote
\usepackage{hyperref}   % permette di aggiungere link cliccabili con \href
\usepackage{tcolorbox}  % cornice attorno alle note
\usepackage{nameref}    % permetti di ottenere il nome di una sezione dal label

\usepackage[outputdir=dist]{minted}  % syntax highlight degli snippet di codice

% per snippet di codice "non compilabile" (deve ignorare whitespace a inizio riga)
\usepackage{listings}
\usepackage{lstautogobble}

\usepackage{hyphenat}  % permette di andare a capo correttamente con le parole
\usepackage{csquotes}  % permette di visualizzare correttamente '<' e '>' (invece di visualizzare '?')
\usepackage{biblatex}  % per citare le fonti attraverso la bibliografia
\usepackage{graphicx}  % permette di inserire figure
\usepackage{float}     % permette di inserire le figure nel punto specificato

% includi i comandi custom
\import{./}{custom_commands.tex}

% le immagini della documentazione sono contenute nella cartella assets
\graphicspath{ {./assets/} }

% Importa il file con la bibliografia
\addbibresource{bibliography.bib}

% rimuovi spazi e tabulazioni a inizio riga nelle sezioni lstlisting
\lstset{
    basicstyle=\ttfamily,
    mathescape=true,
    escapeinside=||,
    autogobble
}


% informazioni pagina di copertina
\author{Stefano Zenaro}
\date{Settembre 2022}
\title{Learning Unreal Engine}


\begin{document}

    % ------------------------------------------------------------------
    %
    % Titolo documento

    \maketitle
    \newpage

    \tableofcontents


    % ------------------------------------------------------------------
    %
    % Setup iniziale
    \import{./}{setup_iniziale.tex}


    % ------------------------------------------------------------------
    %
    % Introduzione
    \import{./}{introduzione.tex}


    % ------------------------------------------------------------------
    %
    % Gameplay Framework
    \import{./}{gameplay_framework.tex}


    % ------------------------------------------------------------------
    %
    % Blueprint
    \import{./}{blueprint.tex}


    % ------------------------------------------------------------------
    %
    % Programmazione C++
    \import{./}{programmazione_cpp.tex}


    % ------------------------------------------------------------------
    %
    % Plugin
    \import{./}{plugin.tex}


    % ------------------------------------------------------------------
    %
    % Annotazioni

    \chapter{Annotazioni}

    \section{Asset}
        Per trovare gli asset e' comodo cercare nel marketplace di Unreal Engine oppure su Quixel.

        SketchFab contiene degli asset gratuiti e a pagamento.

        Un altro sito che contiene invece personaggi e animazioni per i personaggi e' Mixamo.

    \section{Environment Light Mixer}
        Strumento che permette di lavorare con la luce.

    \section{Hotkey utili}

        \begin{itemize}
            \item shift + F1 permette di ottenere il controllo del mouse mentre sta andando il gioco.
            \item tab + ESC permette di fermare il gioco che e' in esecuzione.
            \item alt + P fa partire il gioco.
        \end{itemize}


    % ------------------------------------------------------------------
    %
    % Troubleshooting
    \import{./}{troubleshooting.tex}


    % ------------------------------------------------------------------
    %
    % Bibliografia
    \chapter{Bibliografia}
    \nocite{*}
    \printbibliography[heading=none]

\end{document}
