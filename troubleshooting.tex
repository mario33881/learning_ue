\chapter{Troubleshooting}

    \begin{notebox}
        Questa sezione e' stata realizzata per risolvere problemi con Unreal Engine 5.0.3 e Visual Studio 2022 Community Edition su sistema operativo Windows 10.
    \end{notebox}


    % VISUAL STUDIO
    \section{Visual Studio}

        \begin{nohyphens}
            Se Visual Studio non trova header che esistono oppure si desidera eliminare dei file dal progetto puo' essere utile far ricreare da Unreal Engine i file per Visual Studio: fare click con il tasto destro sul file .uproject e selezionare "Generate Visual Studio project files".
        \end{nohyphens}

        \begin{warningbox}
            Per generare i file e' necessario installare \href{https://dotnet.microsoft.com/en-us/download/dotnet/3.1}{".NET Runtime 3.1.30"}.
        \end{warningbox}

        \begin{suggestionbox}
            Potrebbe essere utile cancellare le cartelle "Binaries", "DerivedDataCache", "Intermediate", "Saved" e il file ".sln" prima di generare i file.
        \end{suggestionbox}


    % PACKAGING
    \section{Packaging}
        I seguenti errori sono relativi al packaging del gioco.

        \subsection{Errore: the sdk for windows is not installed properly}
            Questo errore appare quando non si e' installato \href{https://dotnet.microsoft.com/en-us/download/dotnet/3.1}{".NET Desktop Runtime 3.1.30"} per Windows.


    % FLICKERING OMBRE
    \section{Evitare flickering delle ombre}
        Per evitare il flickering delle ombre bisogna andare in:

        \begin{lstlisting}
            Settings > project settings > engine > rendering
        \end{lstlisting}

        E impostare l'opzione "Shadow Map Method" a "Shadow Maps".

        Un'altra opzione e' impostare la mobility delle luci a "Movable".


    % CRASH DI UNREAL ENGINE
    \section{Crash di Unreal Engine}
        Se Unreal Engine crasha e' molto probabile che si stia cercando di accedere ad un puntatore nullo o invalido.

        Alcune possibili cause sono:
        \begin{itemize}
            \item Non specificare sopra ad un TArray di puntatori "UPROPERTY()": il garbage collector ha eliminato per errore oggetti che non dovevano essere eliminati.

                Aggiungere "UPROPERTY()" permette di dire ad Unreal Engine di gestire la memoria dinamica in modo corretto.

                Se non si aggiunge UPROPERTY si rischia di accedere ad un puntatore che non e' piu' valido facendo crashare Unreal Engine.

            \item Utilizzare Find() su una TMap ed accedere immediatamente al puntatore restituito: Find() restituisce un puntatore nullptr se non e' stato possibile trovare un oggetto con la chiave specificata.

                Accedere al puntatore nullo fa crashare il motore grafico.

            \item Utilizzare NewObject all'interno di un costruttore: NewObject restituira' un puntatore nullptr che, una volta acceduto, fara' crashare l'editor.

                Per risolvere il problema richiamare NewObject in un metodo diverso come, ad esempio, il metodo BeginPlay.
        \end{itemize}


    % ERRORI DI COMPILAZIONE
    \section{Errori di compilazione}

        \subsection{Nested Containers are not supported}
            Questo errore indica che si sta usando UPROPERTY() su un attributo che e' un TArray che contiene un TArray, una TMap che contiene una TMap, \dots E tutte le varie combinazioni di collezioni innestate una dentro l'altra.

            Unreal Engine non supporta la dichiarazione di queste variabili come UPROPERTY().

            Per risolvere il problema e' possibile creare una classe che eredita dal tipo interno e poi creare una collezione di istanze della classe creata.

            \begin{notebox}
                Esempio:

                \begin{minted}[autogobble]{cpp}
                    UPROPERTY()
                    TArray<TArray<int>> mioAttributo;
                \end{minted}

                Questo codice fara' lanciare un errore di compilazione.

                Si potrebbe pensare di risolvere il problema rimuovendo UPROPERTY():

                \begin{minted}[autogobble]{cpp}
                    TArray<TArray<int>> mioAttributo;
                \end{minted}

                Questa soluzione compila ma senza UPROPERTY si rischia che il garbage collector elimini il contenuto dell'array (con "int" NON DOVREBBE succedere ma con oggetti o con puntatori ad oggetti piu' complessi si).

                La soluzione e' creare una classe MioContenitore che eredita da TArray<int>. In questo modo l'attributo sara' di tipo:
                \begin{minted}[autogobble]{cpp}
                    UPROPERTY()
                    TArray<MioContenitore> mioAttributo;
                \end{minted}

                Questa soluzione compila e il contenuto del TArray non sara' eliminato erroneamente dal garbage collector.
            \end{notebox}

            Un'altra soluzione e' creare una collezione di struct/oggetti e ogni struct/oggetto contiene un attributo con la collezione interna.
