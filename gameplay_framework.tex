
\chapter{Gameplay Framework}

    % GIOCATORE
    \section{Giocatore}
        In Unreal Engine il giocatore e' composto da due unita':
        \begin{enumerate}
            \item Pawn: personaggio grafico che interagisce con il mondo di gioco
            \item Controller: interfaccia tra il giocatore reale e il pawn. Gestisce l'interfaccia grafica e in generale l'input dell'utente.
        \end{enumerate}


    % GAME MODE
    \section{Game mode}
        La game mode permette di configurare l'aspetto del pawn e il controller, come inizia un livello, cosa succede quando si smette di giocare\dots E' la prima cosa che viene istanziata quando viene caricata una mappa e quindi e' ideale per fare setup iniziali.

        E' possibile configurare la game mode dell'intero progetto e poi sovrascriverla nei singoli livelli in cui si desidera avere una game mode diversa (se non viene sovrascritta la modalita' di gioco viene ereditata da quella del progetto) dalla finestra:

        \begin{lstlisting}
            Project Settings > Maps & Modes
        \end{lstlisting}

        \begin{notebox}
            In selected mode e' possibile modificare ulteriori informazioni tra cui "default pawn class", ovvero il modello del giocatore.
        \end{notebox}

        Queste informazioni vengono caricate all'inizio di ogni livello e cancellate quando si passa ad un altro livello.

        Per sovrascrivere la game mode bisogna modificare l'impostazine "gamemode override" in:
        \begin{lstlisting}
            window > world settings > game mode
        \end{lstlisting}


    % GAME STATE
    \section{Game state}
        Permette di memorizzare, processare e sincronizzare dati relativi all'interno gioco (come ad esempio un timer che indica quanto tempo manca al termine della partita)

        Queste informazioni vengono caricate all'inizio di ogni livello e cancellate quando si passa ad un altro livello.


    % PLAYER STATE
    \section{Player state}
        Permette di memorizzare, processare e sincronizzare dati relativi al giocatore (ad esempio la vita rimanente, il punteggio, \dots)

        Queste informazioni vengono caricate all'inizio di ogni livello e cancellate quando si passa ad un altro livello.


    % GAME INSTANCE
    \section{Game Instance}
        Sono le informazioni che vengono caricate quando viene aperto il gioco e che vengono cancellate quando viene chiuso il gioco.

        E' utile memorizzare nella game instance tutti i dati che devono essere mantenuti nel passaggio tra un livello ed un altro.


    % MODIFICA COMANDI
    \section{Modifica comandi}
        E' possibile configurare i comandi di controllo (mouse, tastiera, joystick, touch, ...) andando in:

        \begin{lstlisting}
        edit > project settings > engine > input
        \end{lstlisting}

        Ci sono due tipi di binding:
        \begin{itemize}
            \item action: quando si preme un pulsante fa qualcosa. Esempio: tasti della tastiera.
            \item axis: ad ogni frame viene effettuato un qualche check. In genere e' usato per qualcosa di "mobile", come il mouse e i joystick.
        \end{itemize}
